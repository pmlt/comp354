\documentclass{article}
\usepackage{graphicx}
\usepackage[margin=1in]{geometry}
\usepackage{indentfirst}
\usepackage{enumerate}

\begin{document}
\title{Project Testing and Delivery Document}
\author{Team D}
\date{\today}

\maketitle

\vspace*{3.5in}
\begin{table}[htbp]
\caption{Team}
\begin{center}
\begin{tabular}{|r | c|}
\hline
Name & ID Number \\
\hline\hline
Stefanie Lavoie & 1951750 \\
Pinsonn Laverdure & 9684352 \\
Ghislain Ledoux & 6376320 \\
Rigil Malubay & 6262732 \\
Philippe Milot & 9164111 \\
Christopher Mukherjee & 6291929 \\
\hline
\end{tabular}
\end{center}
\end{table}

\pagenumbering{gobble}% Remove page numbers (and reset to 1)
\clearpage

\begin{table}[htbp]
\caption{Revision History}
\begin{center}
\begin{tabular}{|c | c | c | c| }
\hline
Date & Version & Description & Author \\
\hline\hline
30/07/13 & 0.1 & Set up initial layout of deliverable & Philippe Milot \\
\hline
30/07/13 & 0.2 & Finished Section 2.2.1 & Philippe Milot \\
\hline
30/07/13 & 0.28 & Worked on Section 4 (almost complete) & Christopher Mukherjee \\
\hline
30/07/13 & 0.38 & Finished Section 1 & Christopher Mukherjee \\
\hline
04/08/13 & 0.39 & Started Section 3.2 & Stefanie Lavoie \\
\hline
06/08/13 & 0.41 & Finished Section 4 & Christopher Mukherjee \\
\hline
11/08/13 & 0.43 & Added feature of map panel & Rigil Malubay \\
\hline
\end{tabular}
\end{center}
\end{table}

\clearpage

\tableofcontents
\clearpage

\pagenumbering{arabic}% Arabic page numbers (and reset to 1)

\section{Introduction} % COMPLETE

% [The introduction of the document provides an overview of the entire document, briefly introducing what are its goals, and what information is to be found in it.]

This document will describe in detail the testing process and final delivery for the Vessel Monitoring System which was developed in the context of the COMP354 course. This document includes a report on which items were tested and which were not, descriptions of the unit testing and requirements testing that was performed, a description of stress testing that could have been performed (but was \emph{NOT} performed), an installation manual and users manual, and a final cost estimation for the entire project.

\break

\section{Testing Report}

% [This section presents all the testing activities undertaken on the final product, as well as all the individual test cases used.]

\subsection{Test Coverage}

\subsubsection{Tested Items}

% [List all tested items, along with the test cases that were applied on this item. For each test item, explain why it was necessary to test it. For instance, all features listed as requirements for each build is a mandatory test item. In addition, identify at least five units (i.e. classes/methods) and explain why they require unit testing due to their importance in the implementation through their frequency of use and/or the severity of the impact of their misbehavior. You can categorize your tested items, e.g. “Requirements”, “Units”, etc.]

\paragraph{VSF file parsing.} Test that valid VSF files were correctly parsed. Test that invalid VSF files cause a helpful program error to be printed. This is critical, as 

\paragraph{Client-side connection.} Test that the simulator returns a helpful error message when attempting to connect to a non-existing VMS. Test that the simulator correctly connects to an existing VMS.

\paragraph{Time management on simulator side.} Test that the simulator properly respects the \verb|TIMESTEP|, \verb|STARTTIME|, and \verb|TIME| instructions in the VSF file.

\paragraph{Server-side connection.} Test that the VMS can bind to a proper host and port number. Check that the VMS returns a helpful error message when trying to bind to an already-bound host and port number.

\paragraph{Client data validation.} Test that any invalid data coming from a connected client will be silently ignored by the VMS.

\paragraph{VMS Update Rate.} Test that the VMS display is updated as soon as new information comes in. Test that the VMS display is updated at a fixed, regular rate when no data is coming in.

\paragraph{Out-of-range Vessels.} Test that the VMS ignores any vessel located outside its tracking range. Test that the VMS removes any vessel from its list of tracked vessels as soon as it moves out of range.

\paragraph{VMS Table Display.} Test that the VMS displays all possible information about tracked vessels, namely: ID, Type, X-Y coordinates, Speed, Course, Distance from center of radar, last updated time, and risk level.

\paragraph{VMS Table Filtering.} Test that the VMS properly hides rows of vessels with a type unselected in the filter list.

\paragraph{VMS Table Sorting.} Test that all rows are ordered by the currently-selected column in the table.

\paragraph{Map View Display.} Test that the VMS can display the latest received information in a 2D map containing all vessels as dots, with circles around them to indicate high-risk and low-risk ranges.

\paragraph{Map View Filtering.} Test that the VMS properly hides vessels with a type unselected in the filter list.

\paragraph{VMS Access Levels.} Test that the following features are only available when logged-in as administrator: Filtering, Sorting.

\paragraph{Alert Highlighting.} Test that all vessels involved in a low-risk alert are highlighted in yellow in both the table and the map views. Test that all vessels involved in a high-risk alert are highlighted in red in both the table and the map views.

\paragraph{Alert Icon Indicator.} Check that the upper-left icon always indicates the most serious alert level at any given time.

\subsubsection{Untested Items of Interest} % STATUS: COMPLETE

% [List all untested items that you find would necessitate testing. Explain how it could be tested, and why it would be important to test.]

\paragraph{Command-line argument parsing}. We did not write extensive tests for command-line argument parsing, even though it can be considered input to the system (and therefore unsafe). It would be important to test this more thoroughly in the future.

\paragraph{Large client update data}. We did not test the case when client data exceeds 1mb in size (example: an extremely large number of update data sent sequentially in one shot). Theoretically, it would cause the data to be truncated at 1mb, and therefore result in some of the updates to be flagged as ``invalid'' and ignored.

\subsection{Test Cases}

% [Description of all the test cases applied on the tested items using various techniques and testing different aspects of the system. The following sections are mandatory testing perspectives. Other sections can be added to provide appropriate additional testing perspectives. All test cases must be presented as to be reproducible, with the exact data and procedure to convey the test, as well as the expected result.]

\subsubsection{Unit Testing} % Status: COMPLETE

% In your system, identify 2 testable units (classes, modules or subsystems)

% [For each of these two units, include a list of test cases. Explain what techniques were used to derive these tests, e.g. black box/equivalence partitioning, white box/basis path, etc.]

\paragraph{VesselTest}

The \verb|Vessel| class was extensively tested using JUnit. All test cases were designed under the black box principle. The test cases were:

\subparagraph{JavaBean functionality}
Test all basic accessor/mutator methods.

\subparagraph{Coordinate calculation}
Project current coordinates based on last-known coordinates, course, and time elapsed.

\subparagraph{Error conditions}
Test that the class properly throws whenever bad data is supplied.

\paragraph{RadarMonitorTest}

The \verb|RadarMonitor| class was extensively tested using JUnit. All test cases were designed under the black box principle. The test cases were:

\subparagraph{Manual refresh}
Check that when the RadarMonitor is manually refreshed, all its observers are also refreshed.

\subparagraph{Manual update}
Check that when the RadarMonitor is manually fed new data, the data is forwarded to all observers.

\subparagraph{Active alert detection}
Check that when the RadarMonitor is manually fed new data, any alerts triggered by the update are properly detected and observers are notified.

\subparagraph{Passive alert detection}
Check that when vessels passively drift toward each other over time, all relevant alerts are detected and observers are notified.

\subparagraph{Radar range}
Check that when ships drift outside the radar range, they are no longer tracked by the monitor.

\subsubsection{Requirements Testing}

% [For each tested requirement, include a list of test cases presented in the form of a concrete scenario of system usage and expected system reaction.]

\paragraph{Requirement 1: Vessel list displays latest information.}


\subsubsection{Stress Testing} % STATUS: COMPLETE

% Describe potential extreme situations of system usage. Describe the design of tests that would verify system performance under these extreme conditions. Do not perform the tests.

The system was stress-tested with up to 150 vessels tracked simultaneously, using three distinct connected clients. The system has not been proven to be reliable beyond these limits.

\section{System Delivery}

% [This section provides instructions as to how to install and use the software.]

\subsection{Installation Manual}

This manual will explain the procedures that need to be done in order to make the Vessel Monitoring System run on the user's computer. Since Java is a cross-platform language, this program will run on PCs and Mac.

\paragraph{Install Java \\}
\begin{enumerate}[(a)]
  \item By going on the Java website ( http://www.java.com/en/download/ ) you will be able to download the lastest version of Java. It is important that you download at least Java Version 7 for the program to run.
	\begin{figure}[!htb]
	\centering
	\includegraphics[scale=0.55]{images/javaInstall1.jpg}
	\end{figure}
\pagebreak
  \item Once the download is over, run the executable to install Java and follow the on screen instructions.
	\begin{figure}[!htb]
	\centering
	\includegraphics[scale=0.55]{images/javaInstall2.jpg}
	\end{figure}
  \item Once the installation is done, you will get pop up message that says that the installation is complete.
	\begin{figure}[!htb]
	\centering
	\includegraphics[scale=0.55]{images/javaInstall3.jpg}
	\end{figure}
\end{enumerate}

This is all that needs to be done in order to run the Vessel Monitoring System correctly.
\pagebreak
\subsection{Users Manual}
\paragraph{Brief Overview \\ \\}

The Vessel Monitoring System(VMS) is a Java based application that allows its users to check the positions of various kinds of vessels. This application can only be access by current valid users and administrators of the program. To run the program, you will need the two different jar files for the Vessel Monitoring System and the Radar Simulator. The Vessel Monitoring System is the only one that should be selected to run as it will connect itself to the Radar Simulator when necessary.

\paragraph{Login Screen \\ \\}
The Login screen is directly accessed as soon as the VMS program start. A valid password is required to be allowed to use the program. Only two types of users are available: normal user and administrators.% [Not sure if we should just say what the passwords are here??]

	\begin{figure}[!htb]
	\centering
	\includegraphics[scale=0.70]{images/userManual1.jpg}
	\end{figure}

\paragraph{Main Screen \\ \\}
After loging in, the Main Screen will appear. This is where all the important information will be shown. The Main Screen will be different depending on the current user. An administratror will have more options than a normal user. 

	\begin{figure}[!htb]
	\caption{Administrator Main Screen}
	\centering
	\includegraphics[scale=0.35]{images/userManual2_admin.jpg}
	\end{figure}

	\begin{figure}[!htb]
	\caption{Normal User Main Screen}
	\centering
	\includegraphics[scale=0.35]{images/userManual2_user.jpg}
	\end{figure}
\break
\subparagraph{1) Menu \\ \\}

The menu is where the user can, amongst other things, select to logout option or exit the program.
	\begin{figure}[!htb]
	\centering
	\includegraphics[scale=0.70]{images/userManual4.jpg}
	\end{figure}

The most important option of the menu is the "Add New Simulator File..."  because it allows the user to select a specific VSF scenario and run it.

	\begin{figure}[!htb]
	\centering
	\includegraphics[scale=0.70]{images/userManual5.jpg}
	\end{figure}

Here, the host and the port are specific to the machine that is running the Radar Simulator. If both are running on the same machine, than the host should be "localhost" and the port number "11233".

\subparagraph{2) Table View Tab \\ \\}
This is where the user can see a list of all the vessels detected by the radar. The vessels can be ordered in ascending or descending order by clicking the top of a column in the table. The color of a row depends on the type of alert this specific vessel is in.

\subparagraph{3) Map View Tab \\ \\}
This is where the vessels are shown on the map. Different type of vessels have different colors. The position of the vessels will be updated over time. The color of the small circles will change depending on the alert type of a vessel.
	
	\begin{figure}[!htb]
	\caption{Administrator Map View}
	\centering
	\includegraphics[scale=0.36]{images/userManual3_admin.jpg}
	\end{figure}

	\begin{figure}[!htb]
	\caption{Normal User Map View}
	\centering
	\includegraphics[scale=0.36]{images/userManual3_user.jpg}
	\end{figure}

There is a zoom functionality in the map tab, it is enabled by scrolling the mouse forward (Zooming in) and backward (Zooming out) of the the point of origin. To change the point of origin of the zoom, it is simply done by click on the map or by dragging the target to the desired location. The map will only go within 5000 meters of the default origin.
%The user can zoom in and out of the map by using the mouse wheel. 
%Clicking the map will cause it to move around.
\pagebreak
\subparagraph{4) Alerts \\ \\}
The alerts indicates if the vessels are near each other. If the alert circle is green, then all the vessels are far away from each other. If the alert circle is yellow, then this indicates that some vessels are becomimg very close to each other. The circle will turn red when some vessels are dangerously close to each other and an impact might happen.

\subparagraph{5) Filters \\ \\}
This section is only available when the user is an administrator. This allows the user to view only the kind of vessels that are selected.

\break

\section{Final cost estimate} % Status: Work in Progress

% [This shall consist of a table listing all components of all phases of this project, including the person hours cost of each component of each phase. This should include documentation, design, implementation and testing. Include the meeting minutes with action item and the time log as well]

\begin{figure}[!htb]
\caption{Gantt Chart}
\centering
\includegraphics[scale=0.55]{charts/GanttChart.png}
\end{figure}

% Taken from last deliverable with only minor changes. Costs may have to be recalculated.
\begin{description}
  \item[Simulator program:] \$150, cost to code and implement.
  \item[VMS program:] \$150, cost to code and implement.
  \item[Graphical User Interface:] \$100, cost to design, code, and implement.
  \item[Unit test suite:] \$200, cost to write test cases and perform testing.
  \item[Documentation:] \$250, includes all customer deliverables and internal documentation.
  \item[Final integration:] \$90
  \item[Buffer:] \$160
  \item[Total Cost:] \$1100
\end{description}

\paragraph{Basis} Costs have been calculated based on the amount of man hours spent working on project (at a normal approximate labor cost of \$15 an hour). There was no cost associated with tools, because all tools used were either freely available or were provided by the University.

\end{document}
